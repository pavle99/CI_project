\documentclass{article}
\usepackage[utf8]{inputenc}
\usepackage{amsmath, amsfonts, amsthm, graphicx, geometry, lipsum}
\usepackage{hyperref}
\usepackage{algorithm}
\usepackage{algpseudocode}
\usepackage{biblatex}
\usepackage{indentfirst}
\usepackage{listings}
\usepackage[bottom]{footmisc}

\addbibresource{main.bib}

\renewcommand{\contentsname}{Sadržaj}

\makeatletter
\renewcommand{\ALG@name}{Algoritam}
\makeatother

\hypersetup{
    colorlinks=true,
    linkcolor=blue,
    urlcolor=red,
    pdftitle={k-LCS RI},
}

\title{%
    Longest common subsequence korišćenjem Genetskog i Beam search algoritma \\~\\
    \large Projekat iz Računarske inteligencije \\
    Matematički fakultet \\
    Univerzitet u Beogradu
}
\author{
    Pavle Cvejović\\
    \href{mailto:mi18024@alas.matf.bg.ac.rs}{mi18024@alas.matf.bg.ac.rs} \\
    Viktor Novaković\\
    \href{mailto:mi18092@alas.matf.bg.ac.rs}{mi18092@alas.matf.bg.ac.rs} \\
}

\date{Septembar 2022}

\begin{document}

    \renewcommand{\abstractname}{Apstrakt}

    \maketitle

    \begin{abstract}
        Algoritmi na sekvencama\footnote{Koristimo termin \emph{sekvenca} a ne \emph{niska} jer se ovaj problem odnosi na nizove proizvoljnih tipova i ne koristimo termin \emph{niz} jer će se on upotrebljavati u drugom kontekstu}  simbola su izučavani duže vreme i sada formiraju fundamentalni deo računarskih nauka. Jedan od veoma bitnih problema u analiziranju sekvenci je problem pronalaženja najduže zajedničke podsekvence. U generalnom slučaju, kada imamo proizvoljan broj ulaznih sekvenci, ovaj problem je NP-težak \cite{np}. Ovde opisujemo dva pristupa za rešavanje ovog problema - jedan baziran na genetskom algoritmu a drugi na Beam search algoritmu.
    \end{abstract}

    \pagebreak

    \tableofcontents

    \pagebreak


    \section{Uvod}
    Problem pronalaženja najduže zajedničke podsekvence skupa od $k$ sekvenci ($k$-Longest Common Subsequence, $k$-LCS) je jedan od najizučavanijih problema u računarskim naukama u posednjih 30-ak godina jer igra bitnu ulogu u poredjenju sekvenci podataka. Ima potencijalne primene u mnogim područijima. Ovaj problem je koristan u prepoznavanju uzoraka, obradi i kompresiji teksta i podataka \cite{dc} i molekularnoj biologiji \cite{cg}. Može se posmatrati kao \emph{mera bliskosti} $k$ sekvenci jer se sastoji iz pronalaženja najvećeg broja identičnih elemenata svih $k$ sekvenci ali takvih da je bitno uredjenje tih elemenata. Na primer, poredjenje $k$ DNK sekvenci da se vidi njihovo podudaranje ili traženje reči u rečniku blizu pogrešno napisane reči u aplikacijama koje proveravaju pravopis.


    \section{Opis problema}

    Ako imamo dve sekvence $S$ i $T$ na nekoj fiksnoj azbuci $\Sigma$, sekvenca $T$ je \emph{podsekvenca} od $S$ ukoliko se $T$ može dobiti iz $S$ brisanjem nekih elemenata iz $S$. Primetimo da uredjenje preostalih elemenata u $S$ mora biti očuvano. Dužina sekvence $S$ je broj elemenata u njoj i zapisuje se sa $|S|$. Radi jednostavnosti, sa $S[i]$ ćemo obeležavati $i$-ti element u sekvenci $S$, a sa $S[i,j]$ podsekvencu od $S$ koja se sastoji od $i$-tog do $j$-tog elementa iz $S$. Praznu sekvencu obeležavamo sa $\epsilon$.\\

    {\large Problem}: Ako su nam date sekvence $S_i, 1 \leq i \leq k$, na nekoj fiksnoj azbuci $\Sigma$, pronadji sekvencu $T$ koja je podsekvenca $S_i$ za svako $i \in \{1,2,...,k\}$


    \section{Brute force algoritam}

    \subsection{Implementacija}
    Za implementaciju \emph{brute force} algoritma koristili smo tehniku \textbf{dinamičkog programiranja} pronalaženja LCS od k sekvenci. Prvo, podsetimo se kako izgleda funkcija koja prima kao argumente dužine ($n_1, n_2$) 2 sekvence ($a_1,a_2$) i računa dužinu njihovih LCS:
    $$ f(n_1, n_2) =
    \begin{cases}
        0 & \text{ako } n_1 = 0 \lor n_2 = 0 \\
        f(n_1-1,n_2-1)+1 & \text{ako } a_1[n_1 - 1] = a_2[n_2 - 1] \\
        \max{(f(n_1-1,n_2), f(n_1,n_2-1))} & \text{inače}
    \end{cases}
    $$
    Ukoliko bi želeli da generalizujemo ovu funkciju za $k$ sekvenci $a_1,a_2,...,a_k$ sa dužinama $n_1, n_2,...n_k$ dobili bi smo sledeću rekurzivnu formulu:
    $$ f(n_1, n_2, ...,n_k) =
    \begin{cases}
        0, \hspace{1em}  \text{ako je } n_1 = 0 \lor n_2 = 0 \lor \cdots \lor n_k = 0 \\
        f(n_1-1,n_2-1,...,n_k-1)+1, \hspace{1em} \text{ako je } a_1[n_1 - 1] = a_2[n_2 - 1] = \cdots = a_k[n_k-1] \\
        \max{(f(n_1-1,n_2,...,n_k), f(n_1,n_2-1,...,n_k), ..., f(n_1,n_2,...,n_k-1)}, \hspace{1em} \text{inače}
    \end{cases}
    $$

    \vspace{7em}

    Dalje, kreirajmo $k$-dimenzioni niz gde je svaki od nizova iste duzine kao odgovarajuća sekvenca (i inicijalizujmo ih tako da su im svi elementi 0) koji će nam koristiti kao DP tablica za enumeraciju svih sekvenci za računanje puta od LCS. Korišćenjem prethodne formule, možemo jednostavno da izvučemo formulu i za formiranje DP tablice koja će nam pomoći u rekonstrukciji LCS:

    \begin{algorithm}
        \caption{Formiranje DP tablice}
        \hspace*{\algorithmicindent} \textbf{Ulaz: niz sekvenci $a_1,a_2,...,a_k$ i njihovih dužina $n_1, n_2,...n_k$} \\
        \hspace*{\algorithmicindent} \textbf{Izlaz: formirana $k$-dimenziona DP tablica}
        \begin{algorithmic}
            \State \texttt{DP} $\gets \mathbf{0}_{n_1 \times n_2 \times \cdots \times n_k}$
            \For{$(i_1,i_2,...,i_k),(el_1,el_2,...,el_k)$ \texttt{in} \text{enumerate}$(a_1 \times a_2 \times \cdots \times a_k)$}
                \If{$el_1 = el_2 = \cdots = el_k$}
                    \State $DP[i_1, i_2, ... , i_k] \gets DP[i_1-1, i_2-1, ... , i_k-1] + 1$
                \Else
                    \State $DP[i_1, i_2, ... , i_k] \gets \max{(DP[i_1-1, i_2, ... , i_k], DP[i_1, i_2-1, ... , i_k], ..., DP[i_1, i_2, ... , i_k-1])}$
                \EndIf
            \EndFor \\
            \Return \texttt{DP}
        \end{algorithmic}
    \end{algorithm}

    Sada možemo ići "unazad" kroz DP tablicu da rekonstruišemo LCS tako što kada god nadjemo pogodak u tablici nadovežemo element iz bilo koje sekvence (koristeći bilo koji indeks) na rezultujuću podsekvencu:

    \begin{algorithm}
        \caption{Formiranje LCS}
        \hspace*{\algorithmicindent} \textbf{Ulaz: $k$-dimenziona tablica DP, niz sekvenci $a_1,a_2,...,a_k$ i njihovih dužina $n_1, n_2,...n_k$} \\
        \hspace*{\algorithmicindent} \textbf{Izlaz: LCS ulaznih sekvenci}
        \begin{algorithmic}
            \State \texttt{lcs} $\gets$ \texttt{[]}
            \While{$n_i > 0 \hspace{0.5em} (\forall i \in \{1,..,k\})$}
                \State \texttt{step} $\gets DP[n_1, n_2, ..., n_k]$
                \If{$(\exists i \in \{1,...,k\}) \texttt{ step} = DP[n_1, n_2, ...,n_i-1,..., n_k]$}
                    \State $n_i \gets n_i-1$
                \Else
                    \State \texttt{lcs.append($a_1[n_1 - 1]$)}
                    \State $n_i \gets n_i - 1  \hspace{0.5em} (\forall i \in \{1,..,k\})$
                \EndIf
            \EndWhile \\
            \Return \texttt{lcs}
        \end{algorithmic}
    \end{algorithm}

    \pagebreak


    \section{Genetski algoritam}

    \subsection{Uopšteno}
    Pri prvom pristupu problemu, odlučili smo da koristimo genetski algoritam. Kod generacijskog genetskog algoritma, incijalizujemo populaciju sa nasumičnim jedinkama, biramo jedinke roditelje, od njih kreiramo jedinke decu, i sa odredjenom šansom mutiramo dobijene jedinke. Cilj je da pri svakom odabiru roditelja, jedinke sa boljom ocenom imaju veće šanse da budu odabrane. Na ovaj način pokušavamo da imitiramo evoluciju u prirodi.

    \subsection{Predstavljanje jedinki}
    Pretpostavimo bez umanjenja opštosti da je sekvenca najmanje dužine na prvoj poziciji, \emph{i=1}. Tada svaku jedinku u genetskom algoritmu možemo predstaviti kao niz \emph{bulovskih} vrednosti dužine $|S_1|$. Sekvenca koju jedinka predstavlja je sastavljena od karaktera sekvence \emph{$S_1$} za koje u odgovarajućem elementu niza stoji vrednost \emph{Tačno}. Očigledno, ovakva jedinka može da predstavi sve moguće validne sekvence za naš problem.

    \subsection{Fitnes funkcija}
    Ako uzmemo prethodnu pretpostavku, za sekvence $S_1, S_2,..., S_k \in \Sigma^*$ važi $|S_1| \leq |S_i|$ za svako $i \in {2,...,k}$. Neka je $n = |S_1|$ i neka za svaku jedinku $s \in \{0,1\}^n$ \emph{c(s)} bude kandidat za rešenje koji je predstavljen jedinkom $s$, i neka \emph{k(s)} bude broj sekvenci $S_1, S_2,..., S_k$ kojima je \emph{c(s)} podsekvenca.
    Odlučili smo se da izaberemo fitnes funkciju iz rada \cite{ft} koju definišemo kao:
    $$ f(s) =
    \begin{cases}
        3000(|c(s)| + 30k(s) + 50) & \text{ako } |c(s)| = n \land k(s) = k \\
        3000(|c(s)| + 30k(s)) & \text{ako } |c(s)| < n \land k(s) = k \\
        -1000(|c(s)| + 30k(s) + 50)(k - k(s)) & \text{ako } |c(s)| = n \land k(s) < k \\
        -1000(|c(s)| + 30k(s))(k - k(s)) & \text{ako } |c(s)| < n \land k(s) < k \\
    \end{cases}
    $$

    \subsection{Selekcija}
    Za \emph{selekciju} smo koristili turnirsku i ruletsku selekciju raznih veličina. Najbolje se pokazala turnirska selekcija veličine 10. Nasumično se bira 10 jedinki i uzima se najbolja, tj. jedinka medju njima koja ima najveću vrednost fitnes funkcije.

    \subsection{Ukrštanje, mutacija i elitizam}
    Pokretali smo algoritam za veliki broj kombinacija parametera i tipova za ukrštanje, mutaciju i elitizam, i oni koji su se ispostavili da su najbolji iz empirijskih rezultata su \emph{uniformno} ukrštanje (svaki od gena deteta ima jednaku šansu da bude nasledjen od prvog roditelja kao od drugog), \emph{višestruka} mutacija sa verovatnoćom 0.05 (nasumično se bira broj mutiranih gena do 75\% dužine cele jedinke i negira im se vrednost), i elitizam koji cuva 20\% jedinki.

    \subsection{Uslov zaustavljanja}
    Program se zaustavlja ako se dostigne maksimalni broj iteracija (koje zavise od veličine sekvenci i broja sekvenci na ulazu), ili ako jedinka sa najboljim fitnesom ostane nepromenjena u prethodnih nekoliko generacija (ovaj broj takodje zavisi od veličine sekvenci i broja sekvenci na ulazu), ili ako je pronadjeno optimalno rešenje (ovo je samo u slučaju da je pre GA pokrenut i brute force algoritam sa kojim se može uporediti).


    \section{Beam search algoritam}

    \subsection{Uopšteno}
    \emph{Beam search} algoritam se svodi na nepotpunu pretragu stabla u širinu. Skup čvorova, zvani \emph{beam}, se zadržava. Na početku, \emph{beam} sadrži samo koreni čvor. U svakoj iteraciji, čvorovi \emph{beam}-a se šire da bi se dobili njihovi potomci na narednom nivou. Od dobijenih potomaka, čuvaju se $\beta > 0$ sa najboljom ocenom heuristike.

    \subsection{Graf stanja}
    Neka \emph{n} bude maksimalna dužina sekvenci $S_i, i \in {1,...,k}$. \newline
    \indent Neka je $p^L \in \mathbb{N}^k$ vektor prirodnih brojeva takav da važi $1 \leq p_i^L \leq |S_i|$, za $i = {1,...,k}$. Skup $S[p^L]$ definišemo kao $\{S_i[p_i^L, |S_i|] | i = {1,...,k} \}$. Skup $S[p^L]$ se sastoji od uzastopne podsekvence svakog od početnih sekvenca od pozicije iz $p^L$ do kraja. Vektor $p^L$ nazivamo levo-pozicioni vektor, i on predstavlja LCS potproblem na sekvencama $S[p^L]$. \newline
    \indent Graf stanja LCS problema je usmereni aciklički graf \emph{G = (V, E)}, gde je čvor $v \in V$ predstavljen odgovarajućim levo-pozicionim vektorom $p^{L,v}$ i dužinom $l_v$ parcijalnog rešenja. Grana $a = (v_1, v_2) \in E$ sa labelom $l(a) \in \Sigma$ postoji ako je $l_{v_2}$ za jedan veće od $l_{v_1}$, i ako se od parcijalnog rešenja dobijenog dodavanjem karaktera $l(a)$ na parcijalno rešenje čvora $v_1$ dobija potproblem $S[p^{L,v_2}]$. Koreni čvor $r$ grafa stanja je originalni problem, koji se može zapisati kao prazno parcijalno rešenje, tj. $r = (({1,...,1}), 0)$, koje predstavlja praznu sekvencu $\epsilon$. Da bi se dobilo dete čvora $v \in V$, mora se prvo odrediti kojim je karakterima moguće produžiti $v$, tj. karakteri $a \in \Sigma$ koji se pojavljuju makar jednom u svakoj od sekvenci iz $S[p^{L,v}]$. Za svaki od karaktera $a$, neka je pozicija prvog pojavljivanja u $S_i[p_i^{L,v}, |S_i|]$ zapisana kao $p_{i,a}^{L,v}$. Novi čvor dobijen dodavanjem karaktera $a$ na parcijalno rešenje predstavljeno čvorom $v$ je onda $v' = (p^{L,v'}, l_v + 1)$, gde je $p_i^{L,v'} = p_{i,a}^{L,v} + 1$, za svako $i = {1,...,m}$. Listove ovako definisanog stabla nazivamo \emph{kompletnim} čvorovima.\newline
    \indent Takodje uvodimo termin \emph{dominantnog} levo-pozicionog vektora. Levo-pozicioni vektor $p^{L,v_1}$ je \emph{dominantan} nad levo-pozicionim vektorom $p^{L,v_2}$ \textbf{akko} važi $p^{L,v_1}[i] < p^{L,v_2}[i], i = {1,...,k}$.

    \subsection{Implementacija}
    Pre prikaza samog algoritma, potrebno je definisati funkcije koje ćemo definisati unutar njega.\newline
    \indent ExtendAndEvaluate(B, h) generiše sve potomke svakog od čvorova $v \in B$, odredjuje svakom od potomaka vrednost heuristike $h$ i sortira ih neopadajuće po odgovarajućim $h$-$vrednostima$, dobijeni skup čvorova nazivamo $V_{ext}$. Filter($V_{ext}$, $k_{best}$) je jedan od opcionih koraka koji smo odlučili da koristimo. Filter briše sve čvorove iz $V_{ext}$ koji su dominirani od strane drugih čvorova iz $V_{ext}$, ali radi efikasnosti, ne proverava se za svaki od čvorova da li je dominantan nad datim čvorom, nego se za proveru uzimaju najboljih $k_{best}$ iz $V_{ext}$. Reduce($V_{ext}$, $\beta$) vraća novi \emph{beam} koji se sastoji od najbolje ocenjenih $\beta$ čvorova iz $V_{ext}$. Bitno je napomenuti i opcioni korak Prune($V_{ext}$, $ub_{prune}$) koji iz $V_{ext}$ briše sve čvorove $v \in V_{ext}$ za koje važi $l_v + ub_{prune} \leq |s_{lcs}|$, gde je $ub_{prune}$ funkcija koja računa gornju granicu broja karaktera koji mogu da se dodaju na trenutni čvor, a $|s_{lcs}|$ dužina trenutno najboljeg rešenja, ali, radi efikasnosti, odlučili smo da ne koristimo ovaj korak.

    \begin{algorithm}
        \caption{Beam Search}
        \hspace*{\algorithmicindent} \textbf{Ulaz: skup sekvenci $S$ i odgovarajuća azbuka $\Sigma$, funkcija heuristike $h$ za ocenu čvorova, parametar $k_{best}$ za filtriranje, $\beta$ - veličina $beam$-a}\\
        \hspace*{\algorithmicindent} \textbf{Izlaz: Ostvareno LCS rešenje}
        \begin{algorithmic}
            \State \texttt{$B$} $\gets$ \texttt{$\{r\}$}
            \State \texttt{$s_{lcs}$} $\gets$ \texttt{$\epsilon$}
            \While{$B \neq \emptyset $}
                \do \\
                \State \texttt{$V_{ext}$} $\gets$ ExtendAndEvaluate($B$, $h$)
                \State ažuriranje $s_{lcs}$ ako je dostignut \emph{kompletan} čvor $v$ sa novim najvećim $l_v$
                \State \texttt{$V_{ext}$} $\gets$ Filter($V_{ext}$, $k_{best}$)
                \State \texttt{$B$} $\gets$ Reduce($V_{ext}$, $\beta$)
            \EndWhile \\
            \Return \texttt{$s_{lcs}$}
        \end{algorithmic}
    \end{algorithm}

    \subsubsection{Funkcije za ocenjivanje čvorova}
    Fraser \cite{fr} je kao gornju granicu za broj karaktera koji mogu da se dodaju na parcijalno rešenje predstavljeno čvorom $v$, tj. dužinu LCS za potproblem $S[p^{L,v}]$ koristio $\textsc{UB}_{min}(v) = \textsc{UB}_{min}(S[p^{L,v}]) = min_{i={1,...,k}}(|S_i|-p_i^{L,v}+1)$ \newline
    \indent Za funkcije heuristike Mousavi i Tabataba \cite{mt, mt2} predlažu dva pristupa. Prvi pristup se zasniva na pretpostavci da su sekvence problema uniformno nasumično generisane i medjusobno nezavisne. Autori su izveli rekurziju koja odredjuje verovatnoću $\mathcal{P}(p, q)$ da će uniformno nasumična sekvenca dužine $p$ biti podsekvenca uniformno nasumične sekvence dužine $q$. Potrebne verovatnoće mogu da se izračunaju pri pretprocesiranju i čuvaju se u matrici. Koristeći fiksirano $t$, pod pretpostavkom da su sekvence problema nezavisne, svaki čvor se ocenjuje sa $\textsc{H}(v) = \textsc{H}(S[p^{L,v}]) = \prod_{i=1}^k\mathcal{P}(t, |S_i|-p_i^{L,v}+1)$. Ovo odgovara verovatnoći da parcijalno rešenje predstavljeno preko $v$ može da se produži za $t$ karaktera. Vrednost $t$ se heuristički odredjuje kao $t := max(1, \lfloor \frac{1}{|\Sigma|} \cdot min_{v \in V_{ext},i={1,...,k}}(|S_i|-p_i^{L,v}+1) \rfloor)$. Druga heuristička procena je takozvana \emph{power} heuristika:
    $$ \textsc{Pow}(v) = \textsc{Pow}(S[p^{L,v}]) = \left(\prod_{i=1}^k(|S_i|-p_i^{L,v}+1)\right)^q\cdot \textsc{UB}_{min}(v), q \in [0,1).$$
    Gleda se kao generalizovani $\textsc{UB}_{min}$. Autori tvrde da bi trebalo da se uzme manja vrednost $q$ u slučaju većeg $k$ i postavljaju $q = a \times exp (-b\cdot k) + c$, gde su $a, b, c \geq 0$ zavisni od ulaznih parametara.


    \section{Testiranje i rezultati}
    Oba algoritma su pokrenuta na računaru sa sledećim specifikacijama: CPU --- Amd Ryzen 5 1400x @ $4 \times 3.2GHz$, RAM --- 7836MiB,  OS --- Windows 10, python --- Python 3.10.6\footnote{Ime kolone u tabelama sa -Lit. u sufiksu se odnosi na rezultate odgovarajućeg algoritma iz literature, dok se $|\Sigma|$ odnosi na veličinu azbuke, $n$ na veličinu najduže sekvence, $m$ broj sekvenci, $t[s]$ na dužinu izvršavanja i $s_{best}$ na dužinu pronadjene LCS}.

    \subsection{Genetski}
    Pokrenuli smo genetski algoritam da radi sa 108 različitih kombinacija parametara (generation\_size, chromosome\_size, tournament\_size, mutation\_rate, itd...) a prikazujemo rezultate GA dobijene kombinacijom parametara gorespomenutih u tekstu i uporedjujemo sa rezultatima izvršavanja brute force algoritma:

    \begin{center}
        \begin{tabular}{| c c c | c c | c c |}
            \hline
            \multicolumn{3}{|c}{Params} & \multicolumn{2}{|c}{DP} & \multicolumn{2}{|c|}{GA} \\
            \hline
            $|\Sigma|$ & $n$ & $k$ & $t[s]$ & $s_{best}$ & $t[s]$ & $s_{best}$ \\
            \hline
            2          & 10  & 2   & 0.01   & 7          & 0.01   & 7          \\
            \hline
            2          & 20  & 2   & 0.01   & 15         & 0.08   & 15         \\
            \hline
            2          & 10  & 3   & 0.01   & 6          & 0.01   & 6          \\
            \hline
            2          & 20  & 3   & 0.02   & 12         & 0.1    & 12         \\
            \hline
            2          & 10  & 5   & 0.7    & 5          & 0.03   & 5          \\
            \hline
            2          & 20  & 5   & 21.9   & 11         & 0.13   & 11         \\
            \hline
            4          & 10  & 2   & 0.01   & 5          & 0.03   & 5          \\
            \hline
            4          & 20  & 2   & 0.01   & 11         & 0.15   & 10         \\
            \hline
            4          & 10  & 3   & 0.01   & 4          & 0.04   & 4          \\
            \hline
            4          & 20  & 3   & 0.03   & 8          & 0.29   & 7          \\
            \hline
            4          & 10  & 5   & 0.72   & 3          & 0.02   & 3          \\
            \hline
            4          & 20  & 5   & 23.6   & 7          & 0.2    & 7          \\
            \hline
            26         & 10  & 2   & 0.01   & 2          & 0.02   & 2          \\
            \hline
            26         & 20  & 2   & 0.01   & 5          & 0.09   & 4          \\
            \hline
            26         & 10  & 3   & 0.01   & 1          & 0.01   & 1          \\
            \hline
            26         & 20  & 3   & 0.03   & 2          & 0.05   & 2          \\
            \hline
            26         & 10  & 5   & 0.73   & 0          & 0.01   & 0          \\
            \hline
            26         & 20  & 5   & 22.8   & 1          & 0.02   & 1          \\
            \hline
        \end{tabular}
    \end{center}

    \pagebreak

    \subsection{Beam search}
    Pri testiranju pokazalo se da je heuristika $\textsc{Pow}$ brža, ali je zbog toga heuristika $\textsc{H}$ davala bolje rezultate. Za heuristiku $\textsc{H}$ vrednost za $k_{best}$ koja je uzeta je 50, a za $\textsc{Pow}$ je 100. Rezultate za fiksno $n=600$ smo poredili sa \cite{res}:

    \begin{center}
        \begin{tabular}{| c c c | c c | c c | c c | c c |}
            \hline
            \multicolumn{3}{|c}{Params} & \multicolumn{2}{|c|}{BS-H} & \multicolumn{2}{|c|}{BS-Pow} & \multicolumn{2}{c|}{BS-H-Lit.} & \multicolumn{2}{c|}{BS-Pow-Lit.}\\
            \hline
            $|\Sigma|$ & $\beta$ & $k$ & $t[s]$ & $s_{best}$ & $t[s]$ & $s_{best}$ & $t[s]$ & $s_{best}$ & $t[s]$ & $s_{best}$ \\
            \hline
            4          & 50      & 20  & 0.18   & 181        & 0.17   & 173        & 0.04   & 189        & 0.10   & 191        \\
            \hline
            4          & 50      & 100 & 0.21   & 149        & 0.19   & 141        & 0.05   & 158        & 0.09   & 156        \\
            \hline
            4          & 50      & 150 & 0.25   & 147        & 0.21   & 142        & 0.06   & 151        & 0.10   & 150        \\
            \hline
            4          & 50      & 200 & 0.25   & 144        & 0.24   & 140        & 0.07   & 150        & 0.11   & 148        \\
            \hline
            4          & 200     & 20  & 0.51   & 186        & 0.42   & 180        & 0.19   & 191        & 0.29   & 191        \\
            \hline
            4          & 200     & 100 & 0.60   & 155        & 0.51   & 148        & 0.36   & 158        & 0.40   & 158        \\
            \hline
            4          & 200     & 150 & 0.59   & 151        & 0.49   & 143        & 0.26   & 151        & 0.34   & 151        \\
            \hline
            4          & 200     & 200 & 0.69   & 148        & 0.51   & 139        & 0.38   & 150        & 0.38   & 150        \\
            \hline
            4          & 600     & 20  & 1.31   & 192        & 1.28   & 185        & 0.71   & 192        & 1.20   & 191        \\
            \hline
            4          & 600     & 100 & 1.32   & 157        & 1.28   & 150        & 0.68   & 158        & 1.28   & 158        \\
            \hline
            4          & 600     & 150 & 1.45   & 151        & 1.37   & 143        & 1.05   & 152        & 1.43   & 152        \\
            \hline
            4          & 600     & 200 & 1.40   & 150        & 1.29   & 148        & 1.15   & 151        & 1.00   & 150        \\
            \hline
            20         & 50      & 20  & 0.11   & 42         & 0.09   & 36         & 0.08   & 46         & 0.14   & 46         \\
            \hline
            20         & 50      & 100 & 0.19   & 29         & 0.13   & 28         & 0.08   & 31         & 0.13   & 31         \\
            \hline
            20         & 50      & 150 & 0.20   & 28         & 0.12   & 24         & 0.11   & 29         & 0.13   & 29         \\
            \hline
            20         & 50      & 200 & 0.23   & 27         & 0.22   & 24         & 0.11   & 28         & 0.13   & 27         \\
            \hline
            20         & 200     & 20  & 0.65   & 44         & 0.56   & 39         & 0.45   & 47         & 0.50   & 47         \\
            \hline
            20         & 200     & 100 & 0.67   & 32         & 0.54   & 28         & 0.31   & 31         & 0.49   & 31         \\
            \hline
            20         & 200     & 150 & 0.88   & 30         & 0.46   & 26         & 0.46   & 29         & 0.41   & 29         \\
            \hline
            20         & 200     & 200 & 0.88   & 28         & 0.50   & 25         & 0.43   & 28         & 0.47   & 27         \\
            \hline
            20         & 600     & 20  & 1.83   & 45         & 1.66   & 40         & 1.48   & 48         & 1.71   & 47         \\
            \hline
            20         & 600     & 100 & 1.88   & 31         & 1.68   & 25         & 1.20   & 31         & 1.09   & 32         \\
            \hline
            20         & 600     & 150 & 1.94   & 29         & 1.68   & 25         & 1.29   & 29         & 1.66   & 29         \\
            \hline
            20         & 600     & 200 & 2.01   & 27         & 1.70   & 23         & 1.43   & 28         & 1.62   & 28         \\
            \hline
        \end{tabular}
    \end{center}


    \section{Zaključak}

    Na osnovu prethodnih analiza možemo zaključiti da Genetski algoritam uglavnom upadne u lokalni optimum iz koga se ne može izboriti čak ni kroz nekoliko hiljada generacija bez obzira na izbor parametara i tipova operatora dok Beam search algoritam daje bolja rešenja i to u kraćem vremenskom periodu. Takodje možemo primetiti kod Beam search algoritma da su rešenja "visokog kvaliteta" (sa većim parametrom $\beta$) samo ~10\% bolja a vreme izvršavanja je preko 100\% veće.

    \pagebreak

    \printbibliography[heading=bibintoc,title={Literatura}]


\end{document}
